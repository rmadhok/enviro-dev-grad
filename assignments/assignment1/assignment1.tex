% Options for packages loaded elsewhere
\PassOptionsToPackage{unicode}{hyperref}
\PassOptionsToPackage{hyphens}{url}
%
\documentclass[
]{article}
\usepackage{amsmath,amssymb}
\usepackage{iftex}
\ifPDFTeX
  \usepackage[T1]{fontenc}
  \usepackage[utf8]{inputenc}
  \usepackage{textcomp} % provide euro and other symbols
\else % if luatex or xetex
  \usepackage{unicode-math} % this also loads fontspec
  \defaultfontfeatures{Scale=MatchLowercase}
  \defaultfontfeatures[\rmfamily]{Ligatures=TeX,Scale=1}
\fi
\usepackage{lmodern}
\ifPDFTeX\else
  % xetex/luatex font selection
\fi
% Use upquote if available, for straight quotes in verbatim environments
\IfFileExists{upquote.sty}{\usepackage{upquote}}{}
\IfFileExists{microtype.sty}{% use microtype if available
  \usepackage[]{microtype}
  \UseMicrotypeSet[protrusion]{basicmath} % disable protrusion for tt fonts
}{}
\makeatletter
\@ifundefined{KOMAClassName}{% if non-KOMA class
  \IfFileExists{parskip.sty}{%
    \usepackage{parskip}
  }{% else
    \setlength{\parindent}{0pt}
    \setlength{\parskip}{6pt plus 2pt minus 1pt}}
}{% if KOMA class
  \KOMAoptions{parskip=half}}
\makeatother
\usepackage{xcolor}
\usepackage[margin=1in]{geometry}
\usepackage{longtable,booktabs,array}
\usepackage{calc} % for calculating minipage widths
% Correct order of tables after \paragraph or \subparagraph
\usepackage{etoolbox}
\makeatletter
\patchcmd\longtable{\par}{\if@noskipsec\mbox{}\fi\par}{}{}
\makeatother
% Allow footnotes in longtable head/foot
\IfFileExists{footnotehyper.sty}{\usepackage{footnotehyper}}{\usepackage{footnote}}
\makesavenoteenv{longtable}
\usepackage{graphicx}
\makeatletter
\def\maxwidth{\ifdim\Gin@nat@width>\linewidth\linewidth\else\Gin@nat@width\fi}
\def\maxheight{\ifdim\Gin@nat@height>\textheight\textheight\else\Gin@nat@height\fi}
\makeatother
% Scale images if necessary, so that they will not overflow the page
% margins by default, and it is still possible to overwrite the defaults
% using explicit options in \includegraphics[width, height, ...]{}
\setkeys{Gin}{width=\maxwidth,height=\maxheight,keepaspectratio}
% Set default figure placement to htbp
\makeatletter
\def\fps@figure{htbp}
\makeatother
\setlength{\emergencystretch}{3em} % prevent overfull lines
\providecommand{\tightlist}{%
  \setlength{\itemsep}{0pt}\setlength{\parskip}{0pt}}
\setcounter{secnumdepth}{5}

\usecolortheme{beaver}
\useinnertheme{rounded}
\useoutertheme[subsection=false,footline=authortitle]{miniframes}  

\beamertemplatenavigationsymbolsempty
\setbeamertemplate{headline}{}

\setbeamerfont{block title}{size={}}

\definecolor{kdisgreen}{RGB}{0, 99, 52}
\definecolor{kdisplatinum}{RGB}{167, 169, 172}

\setbeamercolor{alerted text}{fg=kdisgreen}
\setbeamercolor{example text}{fg=kdisplatinum}

\setbeamercolor*{palette secondary}{fg=white,bg=kdisplatinum} %subsection
\setbeamercolor*{palette tertiary}{fg=white,bg=kdisgreen} %section

\setbeamercolor{title}{fg=black}         %Title of presentation
\setbeamercolor{section title}{fg=white, bg=kdisgreen}
\setbeamercolor{subsection title}{fg=kdisgreen, bg=kdisplatinum}
\setbeamercolor{frametitle}{fg=kdisgreen, bg=kdisplatinum}

\setbeamerfont{frametitle}{size=\small}
\setbeamerfont{frametitle}{series=\bfseries}

\setbeamertemplate{blocks}[rounded][shadow=true] 
\setbeamertemplate{items}[triangle]
\setbeamercolor{item}{fg=gray, bg=white}

\AtBeginSection{}

\usepackage{tabularx}
\usepackage{changepage}
\usepackage{amsmath}
\usepackage{mathrsfs}
\usepackage{mathtools}
\usepackage{xcolor}
\usepackage{adjustbox,lipsum}
\usepackage{graphicx}
\usepackage{booktabs}
\usepackage{multirow}
\usepackage{appendixnumberbeamer} 
\usepackage{dsfont} 



\ifLuaTeX
  \usepackage{selnolig}  % disable illegal ligatures
\fi
\IfFileExists{bookmark.sty}{\usepackage{bookmark}}{\usepackage{hyperref}}
\IfFileExists{xurl.sty}{\usepackage{xurl}}{} % add URL line breaks if available
\urlstyle{same}
\hypersetup{
  hidelinks,
  pdfcreator={LaTeX via pandoc}}

\title{Assignment 1}
\usepackage{etoolbox}
\makeatletter
\providecommand{\subtitle}[1]{% add subtitle to \maketitle
  \apptocmd{\@title}{\par {\large #1 \par}}{}{}
}
\makeatother
\subtitle{Ph.D.~Applied Microeconometrics\\
KDI School Fall 2023}
\author{}
\date{\vspace{-2.5em}2023-08-16}

\begin{document}
\maketitle

\textbf{Due date: Thursday, September 14th before class}

The goal of this assignment is just to get you coding in R (well, RStudio) and R Markdown. You will use data from the Penn World Tables (information about the latest release \href{https://www.rug.nl/ggdc/productivity/pwt/pwt-releases/pwt100}{\textcolor{kdisgreen}{here}}). I have uploaded an Excel file -- called pwt100 -- to the assignment 1 folder with three tabs:

\begin{itemize}
\tightlist
\item
  Info: You can basically ignore this for our purposes.
\item
  Legend: The variable names in the data are not very informative. This tab tells you what they mean.
\item
  Data: This is is the actual data.
\end{itemize}

You can load Excel data into R with the \texttt{tidyverse} package, specifically with the function \texttt{readexcel} from the \texttt{readxl} package. \href{https://github.com/rstudio/cheatsheets/blob/main/data-import.pdf}{\textcolor{kdisgreen}{Here}} is a handy cheatsheet on importing different types of data into R. Note that you will need to specify which tab in the Excel file you want to load into R!

I would like you to do the following:

\begin{itemize}
\tightlist
\item
  Create a figure with log expenditure-side GDP per capita (per person) on the y-axis and year on the x-axis. How you do this is up to you, but I'd suggest something along the lines of \texttt{geom\_smooth} and \texttt{ggplot2} (you can find a handy cheatsheet for ggplot2 \href{https://www.maths.usyd.edu.au/u/UG/SM/STAT3022/r/current/Misc/data-visualization-2.1.pdf}{\textcolor{kdisgreen}{here}}).
\item
  Create a new variable called ``decade'' that is equal to the decade of the year variable. For example, 1960 would be 1960-1969, 1970 would be 1970-1979, etc. You can do this with the \texttt{cut} function, but there are many other ways to do it, too. Please do 1970 through 2019 (i.e.~1970-1979, 1980-1989, etc.) only. You can drop the other years.

  \begin{itemize}
  \tightlist
  \item
    Create a table with mean expenditure-side GDP p.c., population, and average hours worked \emph{by decade}.
  \end{itemize}
\end{itemize}

Along with the figure and table, I'd like you to include a short write-up describing the results. All of this should be contained in a single document (i.e.~there should be a single document with a figure, a table, and a short write-up describing the results).

For full credit, you must turn in the following:

\begin{itemize}
\tightlist
\item
  R Markdown script
\item
  pdf output of the R Markdown script
\item
  R script if you did not directly do everything in the R Markdown script (I'll leave this choice up to you)
\end{itemize}

All other decisions are up to you. For example, not all countries have data for all years. Do you want to include any available data or drop countries that are missing data? It's up to you. \textbf{Please explain in the write-up what you did and why.} I won't be grading based on right/wrong here, just on whether you can do what I've asked and whether you can explain what you did and why.

\end{document}
